\section{Background}

\subsection{Previous Work on Mental Health of Veterans}

\subsection{Available Database on Sentiment Analysis}

The Big Five Model(BFM) of personality is one of the most popular and efficient
models used in sentiment analysis. First put forward by \cite{bigFiveModel1}
then perfected by \cite{bigFiveModel2} and \cite{bigFiveModel3}, the BFM
consists of Openness, Conscientiousness, Extroversion, Ageeableness, and
Neuroticism. The definitions are showed in Table \ref{table:definitionBFM}.



% TODO try to find a proper definition for the BFM
% These definitions are from Wiki

\begin{table}[b]
  \caption{The definitions of the five factors in BFM}
  \label{table:definitionBFM}
  \centering
  \renewcommand{\tabularxcolumn}{m} % we want center vertical alignment
  \begin{tabularx}{\textwidth}{l >{\raggedright}X}
    \toprule
    \textbf{Name} & \textbf{Features}
    \tabularnewline \midrule
    Openness & The level of intellectually curious, open to emotion, sensitive to beauty and willing to try new things
    \tabularnewline \hline
    Conscientiousness & How people control, regulate, and direct their impulses
    \tabularnewline \hline
    Extroversion & The breadth of activities, surgency from external activity/situations, and energy creation from external means
    \tabularnewline \hline
    Ageeableness & The individual differences in general concern for social harmony
    \tabularnewline \hline
    Neuroticism & The tendency to experience negative emotions, such as anger, anxiety, or depression
    \tabularnewline \bottomrule
  \end{tabularx}
\end{table}

\subsection{Sentiment Analysis on Social Media}

The combination of data mining and machine learning techniques play a important
role in sentiment analysis on social media. There are mainly two approaches -
based on users' social activities and based on linguistic features of
user-generated texts.
