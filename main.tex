\documentclass[english,a4paper,11pt]{article}

\usepackage[utf8]{inputenc}
\usepackage{natbib}
\usepackage{graphicx}  %%% for including graphics
\graphicspath{{./images/}}
\usepackage{url}       %%% for including URLs
\usepackage{times}
\usepackage[margin=25mm]{geometry}

%%% custom packages
\usepackage{booktabs}
\usepackage{tabularx}
\usepackage{hyperref}
\usepackage[autostyle=true]{csquotes}
\usepackage{xpatch}
%%% custom packages

\title{Sentiment Analysis of Soldiers' Tweets - Comparison with civilians (TBC)}
\date{\today}

\author{
  Sumit Mukhija, Rachit Rastogi,\\
  Chao Chen, Chen Wang, Chetan Prasad\\
  School of Computer Science and Statistics, Trinity College Dublin\\
  \texttt{\{mukhijas, rrastogi, chenc1, wangc5, cprasad\}@tcd.ie}
}

\begin{document}
\maketitle
\thispagestyle{empty}
\pagestyle{empty}

\begin{abstract}
  The concern to veterans' mental health should be made. Existing works show that
  mental health changes caused by wars can be reflected in linguistic features of
  the social media texts. In order to detect and compare those changes we collected
  data from 20 soldiers' tweets and examined them with a list of positive and negative
  adjectives to identify the polarity and do a comparison with normal users'
  tweets. The total counts of tweets vary from 57 to 39,000. We identified the
  difference between normal users and soldiers and we did a close look to the
  result with discussion. \\
  \textbf{Keywords:}
\end{abstract}

\section{Introduction}

Microblogging websites are one of the popular online social platforms for people
to express their ideas. And Twitter may be the most popular application in those
platform. Due to the nature of the Twitter messages(tweets), people post their
real time opinions in a particular paradigm, which contains values for
commercial uses and academic researches.

One of the Twitter users' characteristics is the user diversity. In this paper,
we focus on the veterans and soldiers served or serving in wars. We believe wars
can do some significant changes in soldiers' mental health and lead to a negative
psychological state.

We collect public data from APIs by Twitter using a tool available online, then
process and count the words with a list of positive and negatice adjectives to
predict the polarity of the tweets. Then we do the examination on a randomly
collected dataset to compare the difference between veterans/soldiers and the
majority of users.
(TBC due to the experiment implementation)

The rest parts of the paper are organized as follow. In Section 2, we examine on
the literatures related to the topic, with papers related to previous works on
mental health of veterans, available databases on sentiment analysis and
previous works done on sentiment analysis on social media. In Section 3, we
introduce our dataset and the experiment done on the dataset, with the results
we have. In Section 4, we have a deep look into the result and bring the
discussion. In Section 5 and 6, we make conclusion and bring up future works
needed for the topic.

\section{Background}

\subsection{Previous Work on Mental Health of Veterans}

\subsection{Available Database on Sentiment Analysis}

The Big Five Model(BFM) of personality is one of the most popular and efficient
models used in sentiment analysis. First put forward by \cite{bigFiveModel1}
then perfected by \cite{bigFiveModel2} and \cite{bigFiveModel3}, the BFM
consists of Openness, Conscientiousness, Extroversion, Ageeableness, and
Neuroticism. The definitions are showed in Table \ref{table:definitionBFM}.



% TODO try to find a proper definition for the BFM
% These definitions are from Wiki

\begin{table}[b]
  \caption{The definitions of the five factors in BFM}
  \label{table:definitionBFM}
  \centering
  \renewcommand{\tabularxcolumn}{m} % we want center vertical alignment
  \begin{tabularx}{\textwidth}{l >{\raggedright}X}
    \toprule
    \textbf{Name} & \textbf{Features}
    \tabularnewline \midrule
    Openness & The level of intellectually curious, open to emotion, sensitive to beauty and willing to try new things
    \tabularnewline \hline
    Conscientiousness & How people control, regulate, and direct their impulses
    \tabularnewline \hline
    Extroversion & The breadth of activities, surgency from external activity/situations, and energy creation from external means
    \tabularnewline \hline
    Ageeableness & The individual differences in general concern for social harmony
    \tabularnewline \hline
    Neuroticism & The tendency to experience negative emotions, such as anger, anxiety, or depression
    \tabularnewline \bottomrule
  \end{tabularx}
\end{table}

\subsection{Sentiment Analysis on Social Media}

The combination of data mining and machine learning techniques play a important
role in sentiment analysis on social media. There are mainly two approaches -
based on users' social activities and based on linguistic features of
user-generated texts.

\section{Experiment and Results}

\subsection{Experiment Setup}

\subsubsection{Data Collection}

% TODO may need to update the last access date in ref.bib
We use TWINT by \cite{twint} as our data collection tool.
All the data can be accessed publically so there are no ethic considerations.

\subsubsection{Data Process and Analysis}

\subsection{Results}

\input{discussion/discussion.tex}
\input{conclusion/conclusion.tex}
\input{futurework/futurework.tex}

\bibliographystyle{chicago}
\bibliography{ref}

\end{document}
