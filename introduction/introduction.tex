\section{Introduction}

Microblogging websites are one of the popular online social platforms for people
to express their ideas. And Twitter may be the most popular application in those
platforms. Due to the nature of the Twitter messages(tweets), people post their
real time opinions in a particular paradigm, which contains values for
commercial uses and academic researches.

One of the Twitter users' characteristics is user diversity. In this paper,
we focus on the veterans and soldiers served or serving in wars. We believe wars
can do some significant changes in soldiers' mental health and lead to a negative
psychological state.

We collect public data from APIs by Twitter using a tool available online, then
process and count the words with a list of positive and negative adjectives to
predict the polarity of the tweets. Then we do the examination on a randomly
collected dataset to compare the difference between veterans/soldiers and the
majority of users.
(TBC due to the experiment implementation)

The rest parts of the paper are organized as follow. In Section 2, we examine on
the literature related to the topic, with papers related to previous works on the
mental health of veterans, available databases on sentiment analysis and
previous works done on sentiment analysis on social media. In Section 3, we
introduce our dataset and the experiment done on the dataset, with the results
we have. In Section 4, we have a deep look into the result and bring the
discussion. In Section 5 and 6, we make a conclusion and bring up future works
needed for the topic.
